
\documentclass[a4paper]{article}

\usepackage{INTERSPEECH_v2}

\title{Paper Template for INTERSPEECH}
\name{Author Name$^1$, Co-author Name$^2$}
\address{
  $^1$Author Affiliation, Sweden\\
  $^2$Co-author Affiliation, Australia}
\email{author@university.edu, coauthor@company.com}

\begin{document}

\maketitle
% 
\begin{abstract}
  For your paper to be published in the conference proceedings, you must use this document as both an instruction set and as a template into which you can type your own text. If your paper does not conform to the required format, you will be asked to fix it.

  Please do not reuse your past papers as a template. To prepare your paper for submission, please always download a fresh copy of this template from the conference website and please read the format instructions in this template before you use it for your paper.

  Conversion to PDF may cause problems in the resulting PDF or expose problems in your source document. Before submitting your final paper in PDF, check that the format in your paper PDF conforms to this template. Specifically, check the appearance of the title and author block, the appearance of section headings, document margins, column width, column spacing, and other features such as figure numbers, table numbers and equation number. In summary, you must proofread your final paper in PDF before submission.
  
  The maximum number of pages is 5. The 5\textsuperscript{th} page may be used exclusively for references. The references should begin on an earlier page immediately after the Acknowledgements section, and continue onto the 5\textsuperscript{th} page. If no space is available on an earlier page, then the references may begin on the 5\textsuperscript{th} page.

  Index terms should be included as shown below.
\end{abstract}
\noindent\textbf{Index Terms}: speech recognition, human-computer interaction, computational paralinguistics

\section{Introduction}

We approach the problem of  

\section{Methods}

%Overview of experiments, data processing and analysis performed

\subsection{Experiments}

Four tests were conducted for work on age and sex related developmental language differences {Pang...}. Two consisted of children repeatedly producing vocalizations when prompted..., a test where subjects kept there mouths open and did not vocalize... and finally a verb generation task where... The subjects were prompted to produce a response X times for each experiment, and experiments were repeated when.

\subsection{Subjects}

There were 70 (36 boys and 34 girls) healthy native English speaking children who performed the experiment. Their ages ranged from 4.1-18.4 years of age, with a standard deviation of <FILL>. [The children's speech was also verified to show no signs of articulartory difficulties through observation?].

\begin{itemize}
\item Reference in some way to other Pang papers
\item Additional information about handedness and intelligence measures?
\end{itemize}

\subsection{Data Collection and Preparation}

Participants were tested in a magnetically shielded room in the Neuromagnetic Lab at the Hospital for Sick Children, using a CTF 151-channel whole-head MEG system (MEG International Services Ltd., Coquitlam, BC, Canada). The system recorded the 151 MEG channels and a single audio channel with a sampling rate of 4kHz. The MEG signals were resampled at 200Hz, and band pass filtered between 0.5Hz and 100 Hz.

Removal of ocular artifacts was performed... -Rui probably here-

Independent component analysis (ICA) (explain or reference here?) was then used to determine statistically independent sub-components of the MEG recordings across all subjects. This was done by appending MEG recordings for all subjects into a single 151 channel matrix for each of the 4 tests performed using the EEGLAB toolbox {EEGLAB} for Matlab {?}. ICA was performed using EEGLAB's implementation of the logistic infomax ICA algorithm {EEGLAB ICA section reference}.

The weight matrix produced by the ICA for each test condition was applied to each subject's recordings per respective test and the recordings (both MEG and audio separately) were separated into epochs that correspond to -500ms to +1500ms windows around each prompt a subject received to perform a test.

Features were then extracted from all epochs using openSMILE. 

\subsection{Analysis Performed}

The analysis that was performed fell into two categories, in one the features extracted were checked for correlations in an attempt to demonstrate evidence for , in the other models were learned using regularized linear regression to predict 

\section{Results}

\subsection{Correlations}

\subsubsection{MEG vs. Audio Features}

\subsubsection{All Features vs. Age}

\subsection{Regression}

\section{Discussion}

\subsection{Significance}

Frank wants to write section outlining caveats

\section{Conclusions}

Authors must proofread their PDF file prior to submission to ensure it is correct. Authors should not rely on proofreading the Word file. Please proofread the PDF file before it is submitted.

\section{Acknowledgements}

The ISCA Board would like to thank the organizing committees of the past INTERSPEECH conferences for their help and for kindly providing the template files.


\bibliographystyle{IEEEtran}

\bibliography{mybib}

% \begin{thebibliography}{9}
% \bibitem[1]{Davis80-COP}
%   S.\ B.\ Davis and P.\ Mermelstein,
%   ``Comparison of parametric representation for monosyllabic word recognition in continuously spoken sentences,''
%   \textit{IEEE Transactions on Acoustics, Speech and Signal Processing}, vol.~28, no.~4, pp.~357--366, 1980.
% \bibitem[2]{Rabiner89-ATO}
%   L.\ R.\ Rabiner,
%   ``A tutorial on hidden Markov models and selected applications in speech recognition,''
%   \textit{Proceedings of the IEEE}, vol.~77, no.~2, pp.~257-286, 1989.
% \bibitem[3]{Hastie09-TEO}
%   T.\ Hastie, R.\ Tibshirani, and J.\ Friedman,
%   \textit{The Elements of Statistical Learning -- Data Mining, Inference, and Prediction}.
%   New York: Springer, 2009.
% \bibitem[4]{YourName17-XXX}
%   F.\ Lastname1, F.\ Lastname2, and F.\ Lastname3,
%   ``Title of your INTERSPEECH 2017 publication,''
%   in \textit{Interspeech 2017 -- 18\textsuperscript{th} Annual Conference of the International Speech Communication Association, August 20?24, Stockholm, Sweden, Proceedings, Proceedings}, 2017, pp.~100--104.
% \end{thebibliography}

\end{document}
