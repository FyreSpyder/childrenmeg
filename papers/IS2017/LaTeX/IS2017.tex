
\documentclass[a4paper]{article}

\usepackage{INTERSPEECH_v2}

\title{Paper Template for INTERSPEECH}
\name{Author Name$^1$, Co-author Name$^2$}
\address{
  $^1$Author Affiliation, Sweden\\
  $^2$Co-author Affiliation, Australia}
\email{author@university.edu, coauthor@company.com}

\begin{document}

\maketitle
% 
\begin{abstract}

Lorem Ipsum
  
\end{abstract}
\noindent\textbf{Index Terms}: Brain Computer Interfaces, Speech Production

\section{Introduction}

% temporary itemized list of things to cover
\begin{itemize}
\item Goal is building models that relate brain activity to speech production
\item Tie in to Broca's area
\item Tie in to Frank's work with co-articulator measurements/models for speech production??
\item Approach from a machine-learning/data-driven perspective
\end{itemize}  

\section{Methods}

%Overview of experiments, data processing and analysis performed

\subsection{Experiments}

The experiments performed consisted of four tests conducted for work on age and sex related developmental language differences {Pang...}. Two consisted of children repeatedly producing vocalizations when prompted..., a test where subjects kept there mouths open and did not vocalize... and finally a verb generation task where... The subjects were prompted to produce a response X times for each experiment.

\subsection{Subjects}

There were 70 (36 boys and 34 girls) healthy native English speaking children who performed the experiment. Their ages ranged from 4.1-18.4 years of age, with a standard deviation of <FILL>. (The children's speech was also verified to show no signs of articulartory difficulties through observation?).

\begin{itemize}
\item How much more tie in to Pang papers?
\item Additional information about handedness and intelligence measures? Relating to Broca's area typically in dominant hemisphere?
\end{itemize}

\subsection{Data Collection and Preparation}

Participants were tested in a magnetically shielded room in the Neuromagnetic Lab at the Hospital for Sick Children, using a CTF 151-channel whole-head MEG system (MEG International Services Ltd., Coquitlam, BC, Canada). The system recorded the 151 MEG channels and a single audio channel with a sampling rate of 4kHz. The MEG signals were resampled at 200Hz, and band pass filtered between 0.5Hz and 100 Hz.

EOG artifacts are a significant source of MEG signal contamination. Simple rejections of epoched data would result in nontrivial loss of collected information. Removal of electro-ocular (EOG) artifacts was therefore performed using automated Blind Source Separation (BSS) method and examining signal complexities measured by fractal dimensions. Auto-BSS filters EOG artifacts using the SOBI algorithm in AAR's implemention. (TODO: REF)

Independent component analysis (ICA) (explain a little bit, or just reference here?) was then used to determine statistically independent sub-components of the MEG recordings across all subjects. This was done by appending MEG recordings for all subjects into a single 151 channel matrix for each of the 4 tests performed using the EEGLAB toolbox {EEGLAB} for Matlab {?}. ICA was performed using EEGLAB's implementation of the logistic infomax ICA algorithm {EEGLAB ICA section reference}.

The weight matrix produced by the ICA for each test condition was applied to each subject's recordings per respective test and the recordings (both MEG and audio separately) were separated into epochs that correspond to -500ms to +1500ms windows around each prompt a subject received to perform a test.

Features were then extracted from all epochs using openSMILE.

\begin{itemize}
\item 50ms windows with 25ms overlap
\item Number of features for each data mode
\item General feature categories, ie. statistical moments: mean, deviation, kurtosis..., FFT mag, energy,
  LPC, jitter, shimmer...
\item Why did we choose these features? Are we saying kitchen sink approach?
\end{itemize}

\subsection{Analysis Performed}

The analysis that was performed fell into two categories, in the first: features extracted were checked for correlations in an attempt to demonstrate evidence for predictive possibility, in the second: regularized linear regression models were learned to predict a subject's age.

\subsubsection{Correlation}

We used standard Pearson correlation to determine correlation between MEG and Audio features for each of our test conditions. Correlations were analyzed by plotting a map of correlation coefficients with p < 0.01. Additionally we also compared correlation between a combination of both feature sets and subject age.

\subsubsection{Linear Regression}

Three linear regression models to predict age were trained using the \textit{Audio}, \textit{MEG} and \textit{Audio+MEG (Fusion)} datasets respectively. Ten-fold cross-validation was used, such that each dataset was divided into 10 approximately equal groups, and each group had an approximately equivalent and nearly uniform distribution of points from each subject. A held out test set was used to report performance.

Would we put a regression formula here?

\section{Results}

\subsection{Correlations}

Big caveat with all the correlation material is that it represents correlations between ICA transformed data.

\subsubsection{MEG vs. Audio Features}

\begin{itemize}
\item Could use the correlation figures I have been generating.
\end{itemize}


\subsubsection{All Features vs. Age}

Do we want to just list the most correlated features here?

\subsection{Regression}

Table comparing mean and deviation test performance for the three different models.

\section{Discussion}

\subsection{Significance}

Frank wants to write section outlining caveats

\section{Conclusions}

Lorem Ipsum

\section{Acknowledgements}

Lorem Ipsum


\bibliographystyle{IEEEtran}

\bibliography{mybib}

% \begin{thebibliography}{9}
% \bibitem[1]{Davis80-COP}
%   S.\ B.\ Davis and P.\ Mermelstein,
%   ``Comparison of parametric representation for monosyllabic word recognition in continuously spoken sentences,''
%   \textit{IEEE Transactions on Acoustics, Speech and Signal Processing}, vol.~28, no.~4, pp.~357--366, 1980.
% \bibitem[2]{Rabiner89-ATO}
%   L.\ R.\ Rabiner,
%   ``A tutorial on hidden Markov models and selected applications in speech recognition,''
%   \textit{Proceedings of the IEEE}, vol.~77, no.~2, pp.~257-286, 1989.
% \bibitem[3]{Hastie09-TEO}
%   T.\ Hastie, R.\ Tibshirani, and J.\ Friedman,
%   \textit{The Elements of Statistical Learning -- Data Mining, Inference, and Prediction}.
%   New York: Springer, 2009.
% \bibitem[4]{Hastie09-TEO}
%   Gomez-Herrero, G.\ et al,
%   ``Automatic Removal of Ocular Artifacts in the EEG without an EOG Reference Channel,''
%   \textit{Proceedings of the 7th Nordic Signal Processing Symposium}, 2006
% \bibitem[5]{YourName17-XXX}
%   F.\ Lastname1, F.\ Lastname2, and F.\ Lastname3,
%   ``Title of your INTERSPEECH 2017 publication,''
%   in \textit{Interspeech 2017 -- 18\textsuperscript{th} Annual Conference of the International Speech Communication Association, August 20?24, Stockholm, Sweden, Proceedings, Proceedings}, 2017, pp.~100--104.
% \end{thebibliography}

\end{document}
